% Options for packages loaded elsewhere
\PassOptionsToPackage{unicode}{hyperref}
\PassOptionsToPackage{hyphens}{url}
%
\documentclass[
]{article}
\usepackage{amsmath,amssymb}
\usepackage{iftex}
\ifPDFTeX
  \usepackage[T1]{fontenc}
  \usepackage[utf8]{inputenc}
  \usepackage{textcomp} % provide euro and other symbols
\else % if luatex or xetex
  \usepackage{unicode-math} % this also loads fontspec
  \defaultfontfeatures{Scale=MatchLowercase}
  \defaultfontfeatures[\rmfamily]{Ligatures=TeX,Scale=1}
\fi
\usepackage{lmodern}
\ifPDFTeX\else
  % xetex/luatex font selection
\fi
% Use upquote if available, for straight quotes in verbatim environments
\IfFileExists{upquote.sty}{\usepackage{upquote}}{}
\IfFileExists{microtype.sty}{% use microtype if available
  \usepackage[]{microtype}
  \UseMicrotypeSet[protrusion]{basicmath} % disable protrusion for tt fonts
}{}
\makeatletter
\@ifundefined{KOMAClassName}{% if non-KOMA class
  \IfFileExists{parskip.sty}{%
    \usepackage{parskip}
  }{% else
    \setlength{\parindent}{0pt}
    \setlength{\parskip}{6pt plus 2pt minus 1pt}}
}{% if KOMA class
  \KOMAoptions{parskip=half}}
\makeatother
\usepackage{xcolor}
\usepackage[margin=1in]{geometry}
\usepackage{color}
\usepackage{fancyvrb}
\newcommand{\VerbBar}{|}
\newcommand{\VERB}{\Verb[commandchars=\\\{\}]}
\DefineVerbatimEnvironment{Highlighting}{Verbatim}{commandchars=\\\{\}}
% Add ',fontsize=\small' for more characters per line
\usepackage{framed}
\definecolor{shadecolor}{RGB}{248,248,248}
\newenvironment{Shaded}{\begin{snugshade}}{\end{snugshade}}
\newcommand{\AlertTok}[1]{\textcolor[rgb]{0.94,0.16,0.16}{#1}}
\newcommand{\AnnotationTok}[1]{\textcolor[rgb]{0.56,0.35,0.01}{\textbf{\textit{#1}}}}
\newcommand{\AttributeTok}[1]{\textcolor[rgb]{0.13,0.29,0.53}{#1}}
\newcommand{\BaseNTok}[1]{\textcolor[rgb]{0.00,0.00,0.81}{#1}}
\newcommand{\BuiltInTok}[1]{#1}
\newcommand{\CharTok}[1]{\textcolor[rgb]{0.31,0.60,0.02}{#1}}
\newcommand{\CommentTok}[1]{\textcolor[rgb]{0.56,0.35,0.01}{\textit{#1}}}
\newcommand{\CommentVarTok}[1]{\textcolor[rgb]{0.56,0.35,0.01}{\textbf{\textit{#1}}}}
\newcommand{\ConstantTok}[1]{\textcolor[rgb]{0.56,0.35,0.01}{#1}}
\newcommand{\ControlFlowTok}[1]{\textcolor[rgb]{0.13,0.29,0.53}{\textbf{#1}}}
\newcommand{\DataTypeTok}[1]{\textcolor[rgb]{0.13,0.29,0.53}{#1}}
\newcommand{\DecValTok}[1]{\textcolor[rgb]{0.00,0.00,0.81}{#1}}
\newcommand{\DocumentationTok}[1]{\textcolor[rgb]{0.56,0.35,0.01}{\textbf{\textit{#1}}}}
\newcommand{\ErrorTok}[1]{\textcolor[rgb]{0.64,0.00,0.00}{\textbf{#1}}}
\newcommand{\ExtensionTok}[1]{#1}
\newcommand{\FloatTok}[1]{\textcolor[rgb]{0.00,0.00,0.81}{#1}}
\newcommand{\FunctionTok}[1]{\textcolor[rgb]{0.13,0.29,0.53}{\textbf{#1}}}
\newcommand{\ImportTok}[1]{#1}
\newcommand{\InformationTok}[1]{\textcolor[rgb]{0.56,0.35,0.01}{\textbf{\textit{#1}}}}
\newcommand{\KeywordTok}[1]{\textcolor[rgb]{0.13,0.29,0.53}{\textbf{#1}}}
\newcommand{\NormalTok}[1]{#1}
\newcommand{\OperatorTok}[1]{\textcolor[rgb]{0.81,0.36,0.00}{\textbf{#1}}}
\newcommand{\OtherTok}[1]{\textcolor[rgb]{0.56,0.35,0.01}{#1}}
\newcommand{\PreprocessorTok}[1]{\textcolor[rgb]{0.56,0.35,0.01}{\textit{#1}}}
\newcommand{\RegionMarkerTok}[1]{#1}
\newcommand{\SpecialCharTok}[1]{\textcolor[rgb]{0.81,0.36,0.00}{\textbf{#1}}}
\newcommand{\SpecialStringTok}[1]{\textcolor[rgb]{0.31,0.60,0.02}{#1}}
\newcommand{\StringTok}[1]{\textcolor[rgb]{0.31,0.60,0.02}{#1}}
\newcommand{\VariableTok}[1]{\textcolor[rgb]{0.00,0.00,0.00}{#1}}
\newcommand{\VerbatimStringTok}[1]{\textcolor[rgb]{0.31,0.60,0.02}{#1}}
\newcommand{\WarningTok}[1]{\textcolor[rgb]{0.56,0.35,0.01}{\textbf{\textit{#1}}}}
\usepackage{graphicx}
\makeatletter
\def\maxwidth{\ifdim\Gin@nat@width>\linewidth\linewidth\else\Gin@nat@width\fi}
\def\maxheight{\ifdim\Gin@nat@height>\textheight\textheight\else\Gin@nat@height\fi}
\makeatother
% Scale images if necessary, so that they will not overflow the page
% margins by default, and it is still possible to overwrite the defaults
% using explicit options in \includegraphics[width, height, ...]{}
\setkeys{Gin}{width=\maxwidth,height=\maxheight,keepaspectratio}
% Set default figure placement to htbp
\makeatletter
\def\fps@figure{htbp}
\makeatother
\setlength{\emergencystretch}{3em} % prevent overfull lines
\providecommand{\tightlist}{%
  \setlength{\itemsep}{0pt}\setlength{\parskip}{0pt}}
\setcounter{secnumdepth}{-\maxdimen} % remove section numbering
\ifLuaTeX
  \usepackage{selnolig}  % disable illegal ligatures
\fi
\IfFileExists{bookmark.sty}{\usepackage{bookmark}}{\usepackage{hyperref}}
\IfFileExists{xurl.sty}{\usepackage{xurl}}{} % add URL line breaks if available
\urlstyle{same}
\hypersetup{
  pdftitle={REddyProc for Eddy Covariance Data Analysis},
  pdfauthor={Yusri Yusup},
  hidelinks,
  pdfcreator={LaTeX via pandoc}}

\title{REddyProc for Eddy Covariance Data Analysis}
\author{Yusri Yusup}
\date{2023-07-22}

\begin{document}
\maketitle

\hypertarget{step-1-prepare-the-data}{%
\section{Step 1: Prepare the Data}\label{step-1-prepare-the-data}}

\hypertarget{step-1-1-import-the-data}{%
\subsection{Step 1-1: Import the Data}\label{step-1-1-import-the-data}}

The data that will be used is included in the package for demo purposes.
Data: 1. DEGebExample: Gebesee, Germany, data from 2004 to 2006. It is
downloadable at
\url{http://www.europe-fluxdata.eu/home/site-details?id=3} after
registration. 2. Example\_DETha98: Tharandt, Germany, data for the year
1998.

The package wesbite is at
\url{https://bgc.iwww.mpg.de/5622399/REddyProc}

\begin{Shaded}
\begin{Highlighting}[]
\FunctionTok{library}\NormalTok{(REddyProc)}
\FunctionTok{data}\NormalTok{(DEGebExample)}
\FunctionTok{data}\NormalTok{(Example\_DETha98)}
\CommentTok{\# Gebesee, Germany, data }
\FunctionTok{summary}\NormalTok{(DEGebExample)}
\end{Highlighting}
\end{Shaded}

\begin{verbatim}
##     DateTime                        NEE              Ustar       
##  Min.   :2004-01-01 00:30:00   Min.   :-49.919   Min.   :0.0000  
##  1st Qu.:2004-10-01 00:22:30   1st Qu.: -1.864   1st Qu.:0.0640  
##  Median :2005-07-02 00:15:00   Median :  0.635   Median :0.1490  
##  Mean   :2005-07-02 00:15:00   Mean   : -1.935   Mean   :0.1884  
##  3rd Qu.:2006-04-02 00:07:30   3rd Qu.:  1.834   3rd Qu.:0.2800  
##  Max.   :2007-01-01 00:00:00   Max.   : 19.008   Max.   :2.0450  
##                                NA's   :21849     NA's   :1149    
##       Tair               rH               Rg         
##  Min.   :-16.710   Min.   : 15.87   Min.   :   0.00  
##  1st Qu.:  3.360   1st Qu.: 66.61   1st Qu.:   0.00  
##  Median :  9.970   Median : 79.10   Median :   2.04  
##  Mean   :  9.664   Mean   : 75.24   Mean   : 124.71  
##  3rd Qu.: 15.520   3rd Qu.: 87.07   3rd Qu.: 176.03  
##  Max.   : 34.680   Max.   :100.00   Max.   :1046.03  
##                    NA's   :1
\end{verbatim}

\begin{Shaded}
\begin{Highlighting}[]
\CommentTok{\# Tharandt, Germany, data}
\FunctionTok{summary}\NormalTok{(Example\_DETha98)}
\end{Highlighting}
\end{Shaded}

\begin{verbatim}
##       Year           DoY           Hour             NEE         
##  Min.   :1998   Min.   :  1   Min.   : 0.000   Min.   :-34.900  
##  1st Qu.:1998   1st Qu.: 92   1st Qu.: 5.875   1st Qu.: -5.250  
##  Median :1998   Median :183   Median :11.750   Median :  0.360  
##  Mean   :1998   Mean   :183   Mean   :11.750   Mean   : -2.165  
##  3rd Qu.:1998   3rd Qu.:274   3rd Qu.:17.625   3rd Qu.:  2.270  
##  Max.   :1998   Max.   :366   Max.   :23.500   Max.   : 24.700  
##                                                NA's   :6257     
##        LE               H                 Rg             Tair        
##  Min.   :-99.38   Min.   :-199.60   Min.   :  0.0   Min.   :-13.600  
##  1st Qu.:  2.80   1st Qu.: -20.22   1st Qu.:  0.0   1st Qu.:  3.400  
##  Median : 14.22   Median :  -7.66   Median :  0.0   Median :  8.800  
##  Mean   : 36.42   Mean   :  22.32   Mean   :116.5   Mean   :  8.573  
##  3rd Qu.: 48.11   3rd Qu.:  31.07   3rd Qu.:159.5   3rd Qu.: 13.900  
##  Max.   :503.16   Max.   : 590.02   Max.   :996.6   Max.   : 32.800  
##  NA's   :2456     NA's   :2500      NA's   :157     NA's   :85       
##      Tsoil              rH             VPD             Ustar       
##  Min.   :-0.280   Min.   :22.39   Min.   : 0.000   Min.   :0.0100  
##  1st Qu.: 3.970   1st Qu.:63.83   1st Qu.: 0.900   1st Qu.:0.3400  
##  Median : 7.430   Median :78.25   Median : 2.400   Median :0.5600  
##  Mean   : 7.679   Mean   :75.16   Mean   : 3.784   Mean   :0.6107  
##  3rd Qu.:11.775   3rd Qu.:90.45   3rd Qu.: 5.100   3rd Qu.:0.7900  
##  Max.   :19.060   Max.   :99.43   Max.   :34.200   Max.   :8.0300  
##  NA's   :85       NA's   :117
\end{verbatim}

\hypertarget{step-1-2-calculate-needed-paramters}{%
\subsection{Step 1-2: Calculate Needed
Paramters}\label{step-1-2-calculate-needed-paramters}}

Essential parameters can be calculated from available parameters using
function built in REddyProc.

Useful functions: 1. fCalcVPDfromRHandTair(\ldots) 2.
fCalcETfromLE(\ldots) 3. fConvertTimeToPosix(\ldots) 4.
fConvertCtoK(\ldots)

\hypertarget{step-1-3-1-tharandt-dataset-unsupported-timestamp-format}{%
\subsection{Step 1-3-1: Tharandt Dataset: Unsupported Timestamp
Format}\label{step-1-3-1-tharandt-dataset-unsupported-timestamp-format}}

In the Tharandt dataset, the time columns are not suitable for further
analysis in REddyProc. It needs to be converted.

\begin{Shaded}
\begin{Highlighting}[]
\FunctionTok{head}\NormalTok{(Example\_DETha98)}
\end{Highlighting}
\end{Shaded}

\begin{verbatim}
##   Year DoY Hour   NEE   LE      H Rg Tair Tsoil    rH VPD Ustar
## 1 1998   1  0.5 -1.21 1.49 -11.77  0  7.4  4.19 55.27 4.6  0.72
## 2 1998   1  1.0  1.72 3.80 -13.50  0  7.5  4.20 55.95 4.6  0.52
## 3 1998   1  1.5    NA 1.52 -18.30  0  7.1  4.22 57.75 4.3  0.22
## 4 1998   1  2.0    NA 3.94 -17.47  0  6.6  4.23 60.20 3.9  0.20
## 5 1998   1  2.5  2.55 8.30 -21.42  0  6.6  4.22 59.94 3.9  0.33
## 6 1998   1  3.0    NA 1.33 -20.55  0  6.5  4.21 59.25 4.0  0.15
\end{verbatim}

\begin{Shaded}
\begin{Highlighting}[]
\NormalTok{Example\_DETha98V1 }\OtherTok{\textless{}{-}} \FunctionTok{fConvertTimeToPosix}\NormalTok{(Example\_DETha98, }\AttributeTok{TFormat =} \FunctionTok{c}\NormalTok{(}\StringTok{\textquotesingle{}YDH\textquotesingle{}}\NormalTok{),}
                    \AttributeTok{Year =} \StringTok{\textquotesingle{}Year\textquotesingle{}}\NormalTok{,}
                    \AttributeTok{Day =} \StringTok{\textquotesingle{}DoY\textquotesingle{}}\NormalTok{, }
                    \AttributeTok{Hour =} \StringTok{\textquotesingle{}Hour\textquotesingle{}}\NormalTok{)}
\end{Highlighting}
\end{Shaded}

\begin{verbatim}
## Converted time format 'YDH' to POSIX with column name 'DateTime'.
\end{verbatim}

\begin{Shaded}
\begin{Highlighting}[]
\FunctionTok{head}\NormalTok{(Example\_DETha98V1)}
\end{Highlighting}
\end{Shaded}

\begin{verbatim}
##              DateTime Year DoY Hour   NEE   LE      H Rg Tair Tsoil    rH VPD
## 1 1998-01-01 00:30:00 1998   1  0.5 -1.21 1.49 -11.77  0  7.4  4.19 55.27 4.6
## 2 1998-01-01 01:00:00 1998   1  1.0  1.72 3.80 -13.50  0  7.5  4.20 55.95 4.6
## 3 1998-01-01 01:30:00 1998   1  1.5    NA 1.52 -18.30  0  7.1  4.22 57.75 4.3
## 4 1998-01-01 02:00:00 1998   1  2.0    NA 3.94 -17.47  0  6.6  4.23 60.20 3.9
## 5 1998-01-01 02:30:00 1998   1  2.5  2.55 8.30 -21.42  0  6.6  4.22 59.94 3.9
## 6 1998-01-01 03:00:00 1998   1  3.0    NA 1.33 -20.55  0  6.5  4.21 59.25 4.0
##   Ustar
## 1  0.72
## 2  0.52
## 3  0.22
## 4  0.20
## 5  0.33
## 6  0.15
\end{verbatim}

\hypertarget{step-1-3-2-gebasse-dataset-missing-vpd}{%
\subsection{Step 1-3-2: Gebasse Dataset: Missing
VPD}\label{step-1-3-2-gebasse-dataset-missing-vpd}}

Meanwhile, the DEGeb dataset does not have the VPD parameter that could
be useful for gap-filling and partitioning.

\begin{Shaded}
\begin{Highlighting}[]
\FunctionTok{head}\NormalTok{(DEGebExample)}
\end{Highlighting}
\end{Shaded}

\begin{verbatim}
##                  DateTime NEE Ustar  Tair    rH Rg
## 35041 2004-01-01 00:30:00  NA 0.092 -0.06 96.13  0
## 35042 2004-01-01 01:00:00  NA 0.090 -0.14 96.10  0
## 35043 2004-01-01 01:30:00  NA 0.023 -0.16 95.93  0
## 35044 2004-01-01 02:00:00  NA 0.038 -0.17 95.80  0
## 35045 2004-01-01 02:30:00  NA 0.077 -0.19 95.67  0
## 35046 2004-01-01 03:00:00  NA 0.025 -0.23 95.47  0
\end{verbatim}

\begin{Shaded}
\begin{Highlighting}[]
\NormalTok{VPD }\OtherTok{\textless{}{-}} \FunctionTok{fCalcVPDfromRHandTair}\NormalTok{(DEGebExample}\SpecialCharTok{$}\NormalTok{rH, DEGebExample}\SpecialCharTok{$}\NormalTok{Tair)}
\NormalTok{DEGebExampleV1 }\OtherTok{\textless{}{-}} \FunctionTok{cbind}\NormalTok{(DEGebExample,VPD)}
\FunctionTok{head}\NormalTok{(DEGebExampleV1)}
\end{Highlighting}
\end{Shaded}

\begin{verbatim}
##                  DateTime NEE Ustar  Tair    rH Rg       VPD
## 35041 2004-01-01 00:30:00  NA 0.092 -0.06 96.13  0 0.2353394
## 35042 2004-01-01 01:00:00  NA 0.090 -0.14 96.10  0 0.2357827
## 35043 2004-01-01 01:30:00  NA 0.023 -0.16 95.93  0 0.2457012
## 35044 2004-01-01 02:00:00  NA 0.038 -0.17 95.80  0 0.2533640
## 35045 2004-01-01 02:30:00  NA 0.077 -0.19 95.67  0 0.2608249
## 35046 2004-01-01 03:00:00  NA 0.025 -0.23 95.47  0 0.2720758
\end{verbatim}

\hypertarget{step-2-create-reddyproc-object-class}{%
\section{Step 2: Create REddyProc Object
Class}\label{step-2-create-reddyproc-object-class}}

Before REddyProc can work on your data, the data has to be converted to
a REddyProc object.

\hypertarget{tharandt-forest-area-timezone-2}{%
\subsection{Tharandt (Forest Area, Timezone =
+2)}\label{tharandt-forest-area-timezone-2}}

The Anchor Station Tharandt (50.96235°N, 13.56516°E, 385 m a.s.l.) is
located in the Eastern part of a large forested area (60 km²) near the
city of Tharandt, about 25 km SW of Dresden. This forest has a long
history of forestry documentation as well as meteorological and
hydrological measurements. The spruce stand at the Anchor Station
Tharandt was established by seeding in 1887. Besides the dominating
spruce trees some additional tree species are present: in the vicinity
of the measurement tower (500 m ra ????

\hypertarget{gebesee-agricultural-area-timezone-1}{%
\subsection{Gebesee (Agricultural Area, TimeZone =
+1)}\label{gebesee-agricultural-area-timezone-1}}

the study site is located near the village Gebesee about 20 km NW of
Erfurt, the station is located in the middle of an agricultural field,
an approximate \ldots51.0997, 10.9146

\begin{Shaded}
\begin{Highlighting}[]
\NormalTok{?sEddyProc\_initialize  }\CommentTok{\# called by sEddyProc$new}
\NormalTok{?sEddyProc\_sSetLocationInfo}

\CommentTok{\# Tharandt Data}
\NormalTok{EProcDETha }\OtherTok{\textless{}{-}}\NormalTok{ sEddyProc}\SpecialCharTok{$}\FunctionTok{new}\NormalTok{(}\StringTok{\textquotesingle{}DE{-}Tha\textquotesingle{}}\NormalTok{, Example\_DETha98V1, }\FunctionTok{c}\NormalTok{(}\StringTok{\textquotesingle{}NEE\textquotesingle{}}\NormalTok{,}\StringTok{\textquotesingle{}Rg\textquotesingle{}}\NormalTok{,}\StringTok{\textquotesingle{}Tair\textquotesingle{}}\NormalTok{,}\StringTok{\textquotesingle{}VPD\textquotesingle{}}\NormalTok{,}\StringTok{\textquotesingle{}Ustar\textquotesingle{}}\NormalTok{))}
\end{Highlighting}
\end{Shaded}

\begin{verbatim}
## New sEddyProc class for site 'DE-Tha'
\end{verbatim}

\begin{Shaded}
\begin{Highlighting}[]
\NormalTok{EProcDETha}\SpecialCharTok{$}\NormalTok{sLOCATION}
\end{Highlighting}
\end{Shaded}

\begin{verbatim}
## $LatDeg
## [1] NA
## 
## $LongDeg
## [1] NA
## 
## $TimeZoneHour
## [1] NA
\end{verbatim}

\begin{Shaded}
\begin{Highlighting}[]
\DocumentationTok{\#\# Location of Tharandt}
\NormalTok{EProcDETha}\SpecialCharTok{$}\FunctionTok{sSetLocationInfo}\NormalTok{(}\AttributeTok{LatDeg =} \FloatTok{51.0}\NormalTok{, }\AttributeTok{LongDeg =} \FloatTok{13.6}\NormalTok{, }\AttributeTok{TimeZoneHour =} \DecValTok{2}\NormalTok{)}
\NormalTok{EProcDETha}\SpecialCharTok{$}\NormalTok{sLOCATION}
\end{Highlighting}
\end{Shaded}

\begin{verbatim}
## $LatDeg
## [1] 51
## 
## $LongDeg
## [1] 13.6
## 
## $TimeZoneHour
## [1] 2
\end{verbatim}

\begin{Shaded}
\begin{Highlighting}[]
\CommentTok{\# Gebesee Data}
\NormalTok{EProcDEGeb }\OtherTok{\textless{}{-}}\NormalTok{ sEddyProc}\SpecialCharTok{$}\FunctionTok{new}\NormalTok{(}\StringTok{\textquotesingle{}DE{-}Geb\textquotesingle{}}\NormalTok{, DEGebExampleV1, }\FunctionTok{c}\NormalTok{(}\StringTok{\textquotesingle{}NEE\textquotesingle{}}\NormalTok{,}\StringTok{\textquotesingle{}Rg\textquotesingle{}}\NormalTok{,}\StringTok{\textquotesingle{}Tair\textquotesingle{}}\NormalTok{,}\StringTok{\textquotesingle{}VPD\textquotesingle{}}\NormalTok{,}\StringTok{\textquotesingle{}Ustar\textquotesingle{}}\NormalTok{))}
\end{Highlighting}
\end{Shaded}

\begin{verbatim}
## New sEddyProc class for site 'DE-Geb'
\end{verbatim}

\begin{Shaded}
\begin{Highlighting}[]
\NormalTok{EProcDEGeb}\SpecialCharTok{$}\NormalTok{sLOCATION}
\end{Highlighting}
\end{Shaded}

\begin{verbatim}
## $LatDeg
## [1] NA
## 
## $LongDeg
## [1] NA
## 
## $TimeZoneHour
## [1] NA
\end{verbatim}

\begin{Shaded}
\begin{Highlighting}[]
\DocumentationTok{\#\# Location of Gebesee}
\NormalTok{EProcDEGeb}\SpecialCharTok{$}\FunctionTok{sSetLocationInfo}\NormalTok{(}\AttributeTok{LatDeg =} \FloatTok{51.1}\NormalTok{, }\AttributeTok{LongDeg =} \FloatTok{10.9}\NormalTok{, }\AttributeTok{TimeZoneHour =} \DecValTok{1}\NormalTok{)  }
\NormalTok{EProcDEGeb}\SpecialCharTok{$}\NormalTok{sLOCATION}
\end{Highlighting}
\end{Shaded}

\begin{verbatim}
## $LatDeg
## [1] 51.1
## 
## $LongDeg
## [1] 10.9
## 
## $TimeZoneHour
## [1] 1
\end{verbatim}

Note that the \texttt{echo\ =\ FALSE} parameter was added to the code
chunk to prevent printing of the R code that generated the plot.

\end{document}
